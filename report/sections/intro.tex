%%% Local Variables:
%%% mode: latex
%%% TeX-master: "../report"
%%% End:

\subsection{Concurrency in general}
Concurrent programming, opposed to traditional sequential programming
where a computation must be run until termination, concurrent
programming supports execution of several computations interleaving
each other. This paradigm has some strong advantages:
\begin{description}
\item[High responsiveness] \ \\
  As some input becomes available, it can be processed immediately by
  temporarily blocking another thread running on the processor.
\item[Throughput] \ \\
  As we have moved from single core systems, to systems with multiple
  processing units, higher throughput can be achieved if the
  computations are implemented concurrently and run in parallel.
\end{description}

Concurrency is omnipresent within information technology as many
systems rely heavily on it. Examples hereof include internet servers,
(scientific) high performance clusters, and distributed services.

Concurrency also include some risks, and one should be careful
when sharing data between threads. Designing concurrent programs
carelessly can, and likely will, introduce race conditions which can
be harmful and very hard to deal with as they might only occur under
rare circumstances due to a particular sequence of execution. Thus,
debugging can be very hard, as the errors can be close to impossible
to reproduce.

Designs that avoid race conditions take proper care of the access to
shared data. In this assignment we use semaphores and monitors to
achieve mutual exclusion in sections of the code where shared
variables are read or written to.


\subsection{The problem}
This report is about our implementation of a virtual playground with
nine cars, implemented as individual threads, and how we deal with
various concurrent problems. We assume that the reader has read the
description on the course website~\cite{assignment}, and we will
therefore not go further into details regarding the problem in this
section.

For the mandatory extra task, we have chosen to implemented ``EXTRA
(E)'', which is the implementation of the barrier with a variable
threshold using a monitor. This is discussed in the section
\ref{sub:bar-thres}.

Throughout the implementation we have prioritized correctness over
efficiency. This is for instance relevant in our implementation of the
field of the playground (Section~\ref{sec:field}).


\subsection{Structure of the report}
The body of the report is structured such that each section
corresponds to a part of the playground - i.e. the field, the alley,
and the barrier with and without a variable threshold. Each of these
sections contain a subsection describing the design and
implementation, tests and, in a single case, a formal proof.


\subsection{Comment on major versions of the program}
\label{sub:versions}
We have made two major versions of the program. \emph{The first
  version} implements the steps 1 and 3, whereas \emph{the second
  version} also includes steps 4, 5, and ``Extra (E)''.

In the second version we have made the barrier and alley using
monitors, which is also used when solving the extra assignment.
