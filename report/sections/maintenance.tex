%%% Local Variables:
%%% mode: latex
%%% TeX-master: "../report"
%%% End:

Now we will look at Step 5 of the assignment, i.e. car
maintenance. When Shift-clicking a car's gate, the car should be
removed from the field for maintenance. When Control-clicking the gate
again, it should be added at the gate and started, if the gate is
open. Solving this includes the challenges of stopping a thread in a
good manner and making sure that it does not cause problems if it
waiting for or is in the alley. As stated in the assignment, we assume
the barrier is off.


\subsection{Implementation}
\label{sub:main-moni}
When implementing our solution we decided to use exceptions to manage
the execution flow. This means that the endless loop the car thread is
running is inside a try-catch clause. When the thread is interrupted,
we catch the exception and do the clean up. This involves deciding
which semaphores should be signaled or positions on the field that
should be cleared. As we will see in a bit, this varies greatly
depending on the point of execution when the exception occurred.

TODO: thread join

When removing a car for maintenance, the \code{removeCar()} method
from \code{CarControl} is invoked by the event listener. First, the
method checks whether the car is already removed and prints a message
accordingly. In the case where the car is not already being repaired,
it interrupts the car's thread (which means the car, since it extends
\code{java.lang.Thread}), and removes the object for garbage
collection. As the car should be running when being stopped, the
exception occurs in car's \code{run()} method. Here we know that we
should signal the semaphore protecting the car's spot and clear its
spot on the display. Though, there are more.

To keep track of the point of execution we have added a field to the
\code{Car} class called \code{state} which is of the private enum
\code{Car.State}, which can be seen below.

\begin{java}
private enum State {
    INIT,
    WAITING_FOR_SPECIAL,
    WAITING_FOR_POS,
    MOVING,
    ARRIVED,
    FINISHED;
}
\end{java}

The code excerpt~\ref{lst:main-run} shows the outline of the
\code{run()} method, and when the different states are set.

\begin{figure}[H]
  \begin{java}
public void run() {
    // start at gate
    ...

    while (true) {
        try {
            this.state = State.INIT;

            sleep(getSpeed());

            // get next position

            this.state = State.WAITING_FOR_SPECIAL;

            // check if at any significant position
            ...

            this.state = State.WAITING_FOR_POS;

            // wait for new position
            this.getSemaphoreFromPos(newPos).P();

            this.state = State.MOVING;

            // move to new position
            ...

            sleep(getSpeed());

            this.state = State.ARRIVED;

            // clear old position
            ...

            this.state = State.FINISHED;

            // free old position
            this.getSemaphoreFromPos(curPos).V();
            ...

        } catch (InterruptedException e) {
            // if old position not signaled, signal both new and old
            this.repair();
        }
    }
}
  \end{java}
  \caption{Car maintenance solution}
\label{lst:main-run}
\end{figure}

When catching the \code{InterruptedException}, thrown by
\code{interrupt()}, we use the utility method \\
\code{repair()} from the code excerpt~\ref{lst:main-rep}, to do the
cleanup. It works by doing a switch-statement on the state of the car,
as each state represents its own edge case~\footnote{Situation only
  occurring with very specific set of parameters}.

% first signaling the current positions semaphore,
% then conditionally signaling the new positions semaphore. Notice the
% parameter \code{clearNewPos}, setting whether to also signal the new
% position. In the second try-catch clause, it is called with
% \code{repair(curPos != newPos)}. Therefore, it will only signal the
% new position if it has not already been done. In the first try-catch,
% we know the new position's semaphore has not been acquired, so we just
% pass \code{false} as the argument. After signaling the relevant
% semaphores, we need to check if the car is currently in the alley. In
% that case, we have to call \code{alley.leave(this.no)}, to ensure we
% do not mess up the counters. If we do not do this, the alley will have
% registered the car entering, but not leaving. This will cause it to
% think there is always one car in it, so the opposite direction will be
% blocked forever. We check if the car is in the alley with the utility
% method \code{inAlley()}, which can be count in the \code{Car}
% class. Lastly, we stop the thread by calling \code{join()}.

\begin{figure}[H]
  \begin{java}
private void repair() {
    // clean up depending on car's state
    switch (this.state) {
        case INIT:
            cd.clear(curPos);
            this.getSemaphoreFromPos(curPos).V();

            if (inAlley() || atAlleyExit())
                this.alley.leave(this.no);

            break;

        ...

        case FINISHED:
            cd.clear(newPos);
            this.getSemaphoreFromPos(newPos).V();

            if (inAlley() || atAlleyEnterance())
                this.alley.leave(this.no);

            break;

        default:
            break;
    }

    // stop thread
    try {
        this.join();
    } catch (InterruptedException e) {
        cd.println("Exception in Car no. " + no);
        ...
    }
}
  \end{java}
  \caption{Car maintenance solution}
\label{lst:main-rep}  
\end{figure}

We will give a brief outline of each state, and how to cleanup after it:
\begin{description}
\item[\code{INIT}] \ \\
  The car has not done anything other than mark its own position, so
  we clear it and signal its semaphore. If the the is in the alley, or
  at the exit of the alley, we manually call
  \code{alley.leave(no)}. We must do this at the exit, because the
  point of execution has not yet reached its \code{leave} statement.

\item[\code{WAITING\_FOR\_SPECIAL}] \ \\
  We clear its current position and signal its semaphore. The car has also
  reached the point of the method where it tries to enter or exit the
  alley, wait for the barrier etc. This means that we only have to
  leave the alley, as we would have done so already if was at the
  exit or similar.

\item[\code{WAITING\_FOR\_POS}] \ \\
  We clear its current position and signal its semaphore. It has now
  passed the point of execution where it enter and leaves the alley
  etc. so we will not only have to check the car is in the alley, but
  also if is standing the entrance, for which we both have to call
  \code{leave}.

\item[\code{MOVING}] \ \\
  The car is now ``moving'', which means that it is has required the
  semaphore of the new position and marked it self in the middle of
  two positions on the field. We therefore have to clear it
  accordingly and signal both the semaphore for the current and new
  position. We check the alley as in \code{WAITING\_FOR\_POS}.

\item[\code{ARRIVED}] \ \\
  The car has now arrived at its new position, so we clear only the
  new one, but it has not yet signald the old semaphore. Therefore,
  we signal both the new and old one and check the alley as in
  \code{WAITING\_FOR\_POS}.

\item[\code{FINISHED}] \ \\
  The car has now signaled the old semaphore, so we only clear and
  signal the new position's semaphore. Finally, we check the alley as
  in \code{WAITING\_FOR\_POS}.
\end{description}

After having dealt with all the edge cases, we stop the thread by
calling \code{join()}.


\subsection{Testing}
\label{sub:main-test}
As we have seen, car maintenance presents many edge cases which also
makes it a hassle to confirm that the code works. If one makes a
change to the code and wants to check that it did not break anything,
it will have to manually check all possible scenarios to deal with
all the paths of execution shown above.

We have therefore tried to make this process more practical and
efficient by creating some tests to recreate specific scenarios. As
due to challenges with timing, we have created two randomized tests
which then needs to be run in succession til one is satisfied that the
code works.

When doing randomized tests, we create a random number generator based
on a randomly generated seed~\footnote{A number used to initialize a
  random number generator}. This seed is printed so the test will be
\emph{reproducible}, in case of a failure.

We will now describe some specific edge cases which can be tested.

First, it is the obvious test of trying to remove and restore a car in
rapid succession. As this is unpractical to test manually, as a person
can only perform the tasks at a moderate speed. Programmaticaly, we
can do this as fast as the system allows it. Therefore, test 12
generates a random number $n$ between zero and nine, which is the car
to be tested. Then it removes and restored car number $n$ ten times in
rapid succession.

The second test we have created to test car maintenance is a stress
test. It iterates a hundred times, for each iteration taking a random
car. The random car is removed and after waiting a second, restored
again. If this test is run multiple times, one becomes more and more
confident that all edge cases are tested, and that the code works as
intended. We have run this tests with several thousands of successive
iterations without failure.
