%%% Local Variables:
%%% mode: latex
%%% TeX-master: "../report"
%%% End:

In this section we will discuss our implementations of the different
features.

\subsection{Starvation in the alley}
There is a potential for starvation if the children going in one
direction decide to bully the others and always keep at least a single
car in the alley, while the others have return to the alley entrance.

This is similar to what the slowdown does, as cars outside the alley
drive fast to get into the alley again. Thus, the cars waiting to
enter have to wait longer.

Fortunately, and due to the variation of the speeds of the cars, there
is not starvation, as the cars will at some point get lucky and
progress. There is, however, a lot of unnecessary waiting.

\subsection{Semaphores versus Monitors}
\label{sub:disc-sema-moni}
Using semaphores allows for finely tuned synchronization, but
sometimes at the cost of additional complexity compared to
implementing the same functionality using monitors.

As aspiring software engineers we highly value code that is easy to
comprehend and maintain, so we would prefer to sacrifice some overhead
to achieve a better code base.

\subsubsection{Field}
\label{subsub:disc-field}
For our implementation of the field, we believe that using semaphores
is preferable. This is due to the simplicity of calling \code{P()} to
request a new position on the map, and \code{V()} to free the
previously held.

\subsubsection{Alley}
\label{subsub:disc-alley}
Regarding the alley, we find the monitor-implementation easier to
comprehend, and thus likely less error prone when being
implemented. From a development perspective, this is definitely to be
preferred.

\subsubsection{Barrier}
\label{subsub:disc-barrier}
Regarding the barrier, we find the two basic (i.e. no variable
threshold) implementations equally complex. When we chose to implement
the variable threshold, we did it using the monitor as it seemed
simpler, as we could avoid calling \code{P()} and \code{V()} on the
three semaphores several times just to get proper synchronization.
