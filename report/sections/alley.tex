%%% Local Variables:
%%% mode: latex
%%% TeX-master: "../report"
%%% End:

The alley is the narrow path on the left side of the board, where
there can fit no more than one car side-by-side. As we have solved the
challenge with collisions, if two cars were to go into the alley, one
from each side, they would create a deadlock as neither one of them
can drive through the other. To solve this we need a way of telling if
there are already cars in the alley, traveling the opposite
direction. If that is the case, then the car should wait until the
alley is free of cars, and then drive in. Of course, the cars waiting
for the alley to become free must wait in a spot where they are not
blocking the cars coming out in the opposite direction.


\subsection{Semaphore implementation}
\label{sub:all-sema}
In the design of our alley implementation, we have drawn heavy inspiration from the
Reader-Writer-Problem~\cite[p. 170]{andrews}, i.e. we have to protect
both reading and writing with mutual exclusion.

In our implementation we have two variables which need to be
protected; \code{upCount} and \code{downCount} which counts how many
cars are going up or down, respectively. In order to protect both
these variables we have the two semaphores \code{mutexUp} and
\code{mutexDown}, while also having the semaphore protecting the alley
from being entered from both ways, \code{alleyFree}.

When a car wants to enter the alley, it must first wait for the
\code{mutex} semaphore, before incrementing the counter. It then checks if the counter
equals one, which means the car is the first one waiting and it must wait
for the \code{alleyFree} semaphore. Upon receival, it signals the
\code{mutex}, such that others cars going the same direction can enter too, and enters the alley.

Below we have an excerpt from the Java implementation.

\begin{figure}[H]
\label{lst:all-sem}
  \begin{java}
public void enter(int no) throws InterruptedException {
  // get direction of car
  AlleyDirection dir = (no < 5) ? AlleyDirection.UP : AlleyDirection.DOWN;

  if (dir == AlleyDirection.UP) {
    this.mutexUp.P();

    this.upCount++;
    // if first one waiting, wait for alleyFree
    if (this.upCount == 1)
      this.alleyFree.P();

    this.mutexUp.V();

  } else {
    // same, only with downCount
  }
}
  \end{java}
  \caption{Semaphore solution from \code{Alley.java}}
\end{figure}

On leaving the alley, the same pattern of steps is made. It waits for
the \code{mutex}, \emph{decrements} the counter, sees if is the last
one leaving, and in that case, signals \code{alleyFree}. The
implementation can be found in the \code{leave}-function, in
\code{Alley.java}.


\subsection{Monitor implementation}
\label{sub:all-moni}
The monitor implementation utilizes the \code{wait} and \code{notify}
methods of \code{java.lang.Object}. This allows for easy signaling
between threads, where we only have the \code{upCount} and
\code{downCount} variables. Therefore, instead of waiting for the
semaphore to be signaled, it calls the \code{wait} method on the
thread.

More specifically, when a car tries to enter the alley it waits
(calling the \code{wait} method) until the \code{count} variable for
the opposite direction is zero, meaning there are no cars in the alley
going the opposite direction. It then increments the \code{count}
variable for its direction and enters the alley.

\begin{figure}[H]
\label{lst:all-mon}
  \begin{java}
public synchronized void enter(int no) throws InterruptedException {
    // get car direction
    AlleyDirection dir = (no < 5) ? AlleyDirection.UP : AlleyDirection.DOWN;

    if (dir == AlleyDirection.UP) {
        // wait while there are cars going the opposite direction
        while (this.downCount > 0)
            wait();

        this.upCount++;

    } else {
        // same, but opposite variables
    }
}
  \end{java}
  \caption{Monitor solution from \code{AlleyMonitor.java}}
\end{figure}

When trying to leave we again follow the same pattern of steps, only
in the opposite order. The car starts with decrementing the
\code{count} variable and then notifying the waiting car by calling
\code{notify}.

We actually use \code{notifyAll}, and not \code{notify}, to make sure
we notify the correct car (as there is at most one car waiting). This
could be optimized, but we prioritize correctness over
performance. Also, Java ``suffers'' from spurious wake ups of waiting
threads, so one could question whether the use of \code{notify}
instead of \code{notifyAll} would have a significant effect.

\subsection{Liveness and safety properties}
\label{sub:all-proof}
\subsubsection{Safety}
We want to avoid cars going in both directions to simultaneously enter the alley, as this will cause a deadlock. Thus we can state an invariant which must always hold:


\emph{Whenever one or more cars heading in one direction are inside the alley, cars driving the opposite direction may not enter}


We can use model-checking in Spin to verify that our program actually adhere to this invariant.

Our model contains the cars (\code{proctypes}), semaphores, the functions \code{enter()} and \code{leave()}, as well as variables used to check that the invariant is adhered to.

The model works by having 4 cars driving upwards, and 4 driving downwards. When a car (a process) has entered the alley, all it does is check that the invariant holds, then leaves, and repeats. As Spin checks all interleavings, this mean that even though it seems that a car instantly leaves the alley, Spin also check the cases where other cars call \code{enter()} while this car is still inside the alley.

The downwards headed cars are implemented as follows:
\begin{promela}
active [ndown] proctype car_down(){
	do::
		enter_down();
		// The car is now inside the alley
		assert(proof_up_in_alley == 0 && proof_down_in_alley > 0);
		leave_down();
	od
}
\end{promela}

The upwards going is implemented symmetrical to this.

As the Java-implementations of the functions \code{enter()} and \code{leave()} are split into two cases, up/down, we have done the same for our Spin-implementation. Our Spin-implementation of these functions are identical to the ones in Java, except we have added a single statement keeping track of how many cars have entered the alley, and the direction they are headed.

This is what is used in the car \code{proctype} to check that the invariant hold when a (downwards headed) car is inside the alley:
\begin{promela}
assert(proof_up_in_alley == 0 && proof_down_in_alley > 0);
\end{promela}



Even though the check is only done when at least one car is inside the alley, the invariant always hold, as it trivially holds when there is no car inside the alley.


\subsubsection{Liveness}
We cannot really prove this, but as a few rounds of execution will reveal that cars in queue to enter the alley will eventually do so. This is also helped by the fact that the speed of the cars vary with each round, such that the cars with access to the alley will most likely not continue to keep the access forever. Thus, the queued cars will not starve.


\subsection{Testing}
\label{sub:all-test}
As our proof in Promela only verifies that no cars driving the
opposite direction of those already in the alley can enter, they might
still attempt.

This can visually be tested by running any test, as they all drive through
the alley, or simply by starting all the cars.
