%%% Local Variables:
%%% mode: latex
%%% TeX-master: "../report"
%%% End:

The field is implementation of the actual ``board'', where the cars move
around. To avoid cars colliding together when trying to move to the
same spot on the board, we use semaphores. In principle, a car waits
for the next spot it plans to move to. Only first when the car gets
it, it signals its previous spot's semaphore. This way we can be
certain that there is no race-conditions or alike when trying to move
from one spot to another.


\subsection{Implementation}
\label{sub:field-impl}
A simple way of implementing the field with samaphores would be to
have a two-dimentional array, with the same sizes as the field, with
semaphores. This means, one semaphore per spot on the
board. Therefore, no spot can be occupied by more than one car, which
makes it impossible for the cars to collide.

Although a simple solution, it is not very effiecient. One can imagine
a solution where there are only as many semaphores in total, as there
are cars. This, though, causes great challenges when deciding on the
correct semaphore to wait on. A challange which we have chosen not to
prioritize in our solution.


\subsection{Testing}
\label{sub:field-test}
It is fairly simple to test whether collitions occurs. First, turn on
the ``keep crash'' option in the GUI, to make it easier to see if a
crash occurs as the spot will be marked with a red dot.

Now, one can simply start all the cars and observe, or run test number
1. All this test does is to greatly increase the speed of cars 1 and 8
(one in each direction). This forces them to constantly bump into the
car infront, as it is traveling much slower.
