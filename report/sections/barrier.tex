%%% Local Variables:
%%% mode: latex
%%% TeX-master: "../report"
%%% End:

The basic version of the barrier is there to ensure fairness among the
children such that no one gets to drive more rounds than the
others. Thus they have to wait next to their starting positions until
the remaining children arrive. In the version with the variable
threshold, the children only have to wait until a certain number of
other cars arrive.

Our barrier works by having two counters that have to reach the
threshold before the cars are released from it. Having only one such
counter, one car could rush to the barrier before all of the other
cars were even released. But by implementing two, that have to reach
their threshold before releasing all their cars, rushing ahead to the
same point, and thus stealing a ``release-token'' from one of the
other cars, is impossible.


\subsection{Implementation with semaphores}
\label{sub:bar-sema}
We have implemented the barrier using the three primary functions;
\code{sync()}, \code{on()}, and \code{off()} as well as an auxiliary
function \code{free()}.

The snippet below shows how one of the two counters is implemented in
the function \code{sync()}. Obviously, shared variables are enclosed
within the statements \code{mutex.P()} and \code{mutex.V()} to ensure
mutual exclusion.

If a car arrives, and the counter is below the threshold, it calls
\code{barrierIncoming.P()} and waits until it gets signaled X
times. When the counter reaches the threshold,
\code{barrierIncoming.V()} is called X times.

The same happens for the semaphore \code{barrierLeaving} in the part
of the code that has been left out.

\begin{figure}[H]
\begin{java}
public void sync() throws Exception {
	this.mutex.P();

	if (this.active) {
		// 1st - all cars must arrive ("incoming")
		this.incomingCount++;

		if (this.incomingCount < this.threshold) { // All cars, except one
			this.mutex.V();
			this.barrierIncoming.P();
		} else { // Final car needed to start a new round
			this.free(this.incomingCount - 1, BarrierSelector.INCOMING);
			// signals the relevant barrier-semaphore (leavingCount-1) times
			this.incomingCount = 0;
			this.mutex.V();
		}
		// 2nd - all cars must leave ("leaving")
		// NB: PART LEFT OUT (This part is symmetrical to the first part)
		// Please see appendices for the full code
	} else {
		this.mutex.V();
	}
}
\end{java}
\caption{Semaphore solution from \code{Barrier} class}
\label{lst:bar-sema}
\end{figure}


When turning on the barrier, the function \code{on()} is called, and
it simply checks if the barrier is not already enabled, and if it is
not, it is turned on, and the counters are set to zero.

Switching off the barrier is a bit more interesting, as the function
\code{off()} is as followes (enclosed in a try-catch block and with
mutual using the \code{mutex}-semaphore):

\begin{java}
if (this.active) {
	this.active = false;
	this.free(this.incomingCount, BarrierSelector.INCOMING);
	this.free(this.leavingCount + this.incomingCount, BarrierSelector.LEAVING);
	this.incomingCount = 0;
	this.leavingCount = 0;
}
\end{java}

When turning off the barrier, the semaphore for the ``arriving'' cars
is signaled as many times as arriving cars has been counted - simply
to release all of them. If there are any arriving cars, these will
then move onto the ``leaving'' section \code{sync()}, and should be
accounted for. This is why the semaphore for the leaving cars is
signaled (\code{incomingCount} + \code{leavingCount}) times. Finally,
the counters are reset.

\subsection{Implementation with a monitor}
\label{sub:bar-moni}
The implementation of the barrier by using a monitor, makes use of the
same idea as the one with semaphores, i.e. using two counters to make
sure that all 9 cars get exactly one turn each.

The methods \code{sync()}, \code{on()} and \code{off()} are all
\code{synchronized}, so mutual exclusion is implemented in a, at least
in our opinion, more simple and comprehensive manner.

\begin{figure}[H]
\begin{java}
public synchronized void sync() {
	if (!this.active)
		return;

	// 1st - all cars must arrive ("incoming")
	// Wait until the barrier accepts incoming cars
	this.waitForMode(BarrierSelector.INCOMING);
	if(!this.active) //Check if barrier is still active when awoken
		return;

	this.incomingCount++;

	if(this.incomingCount == this.threshold) {
		this.mode = BarrierSelector.LEAVING;
		this.incomingCount = 0;
		notifyAll();
	}	// 2nd - all cars must leave ("leaving")
	// NB: PART LEFT OUT (This part is symmetrical to the first part)
	// Please see appendices for the full code
}
\end{java}
\caption{Monitor solution from \code{BarrierMonitor} class}
\label{lst:bar-moni}
\end{figure}

Firstly, the function checks if the barrier is even activated and
returns if not.

Secondly, it calls the auxiliary function
\code{waitForMode(selector)}. This function is implemented as follows:

\begin{figure}[H]
\begin{java}
private void waitForMode(BarrierSelector selector) {
	// Wait until the barrier accepts the wanted type of cars
	while (this.mode != selector) {
		try {
			if(!this.active) //Check if barrier is still active when awoken
				return;
			else
				wait();
		}
		catch (InterruptedException ex) { return; }
	}
}
\end{java}
\caption{\code{waitForMode(selector) from \code{BarrierMonitor} class}}
\label{lst:bar-wait}
\end{figure}

This function simply makes the caller of \code{sync()} wait until the
barrier either accepts the requested type of cars \footnote{incoming
  or leaving as described in section \ref{sub:bar-sema}} or the
barrier is turned off.

When the thread is done waiting for the correct flag, or until the
barrier is turned off, it checks - again - if the barrier is active
and increments the \code{incomingCount}.

If the incoming car is the last one to arrive, the barrier releases
the cars to move on to the ``leaving'' section of the function. The
leaving-section is symmetrical to the one just described, and when
\code{leavingCount} reaches the threshold, the cars are free to take a
new round on the playground.

Regarding \code{on()} and \code{off()}, they are almost identical to
the ones described in section \ref{sub:bar-sema}, except that
\code{off()} sets the mode to \code{BarrierSelector.INCOMING} and
simply calls \code{notifyAll()} instead of an auxiliary function we
have implemented ourselves.


\subsection{Implementation with a variable threshold}
\label{sub:bar-thres}
The implementation of the variable threshold, should take care of
several different cases:

\begin{enumerate}
\item When \code{k} is lower than the current threshold and lower than
  or equal to the number of cars waiting. In this case all cars should
  be released.
\item When \code{k} is lower than the current threshold, but greater
  than the current number of waiting cars, the threshold should simply
  be lowered, but no cars released until the new threshold is reached.
\item When \code{k} is higher that the current threshold, the calling
  thread has to be blocked until all cars have left and then increase
  the threshold. Or if the barrier is turned off while waiting for the
  cars to reach the threshold, then it is also increased.
\item Finally, when \code{k} is equal to the current threshold, we
  simply ignore it and return without modifying anything.
\end{enumerate}

Before checking which case the value of \code{k} fits into, the
function, \code{setThreshold(k)}, actually waits for currently leaving
cars to actually have left the barrier. If this check is not
performed, and the threshold was changed while cars were leaving,
unwanted behavior might occur. As the threads (/cars) do not even call
\code{sleep()} in between the two steps of the barrier, we think the
time spent waiting for the cars to leave, is negligible for all
practical purposes. Thus the four cases takes care of the specified
functionality.

\emph{The first case} is handled as follows:
\begin{java}
this.mode = BarrierSelector.LEAVING;
this.leavingCount = k - this.incomingCount;
this.incomingCount = 0;
this.threshold = k;
notifyAll();
\end{java}

By setting the mode to ``\code{LEAVING}'', all cars that have arrived
will, when awoken, leave the barrier.

The seemingly odd choice of setting \code{leavingCount} to a value
less than zero, does actually make sense, as when all cars have left,
\code{leavingCount} will be equal to \code{k}, which by then also
happens to be the threshold. And then the mode of barrier will be set
to ``\code{INCOMING}''.

\emph{The second case} is handled simply by setting the threshold to
the value of \code{k}, as \code{incomingCount} will still be less than
this value, and nothing unexpected will occur.

\emph{In the third case} the flag \code{increaseThreshold} is set to
the value of \code{k}. When the final car is leaving with the old
threshold still in effect, the threshold is lowered. Also, the caller
of \code{barrierSet(k)} is blocked until either the cars have left the
barrier or the barrier is deactivated.


\subsection{Testing}
\label{sub:bar-test}
The barrier can be tested with several tests, implemented in
\code{CarTest.java}. Firstly, there are tests two through nine which simply
sets the barrier threshold to the number of the test, starts the same number of cars, minus
one, before waiting a little and then starting the last one. This
clearly demonstrates that the barrier does not open before the threshold is reached.

We have also made tests for variable threshold. Test 10 starts all
cars, sets the barrier threshold to four, then wait a little before
setting the threshold up to nine. Here one can see that the barrier first
waits for four cars, before waiting for nine cars after. The test is
shown in code excerpt~\ref{lst:bar-test}.

\begin{figure}[H]
\begin{java}
case 10:
    log("variable barrier threshold (low to high)");

    setSpeedAll(100);

    cars.startAll();
    cars.startCar(0);
    Thread.sleep(1000);

    cars.barrierOn();
    t = 4;
    log("threshold " + t);
    cars.barrierSet(t);
    Thread.sleep(2000);

    t = 9;
    log("threshold " + t);
    cars.barrierSet(t);
    break;
\end{java}
\caption{Test 10 - variable threshold}
\label{lst:bar-test}
\end{figure}

Test 11 does the same, only it starts by setting it to nine, then sets
it down to four.
