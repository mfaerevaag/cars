%%% Local Variables:
%%% mode: latex
%%% TeX-master: "../report"
%%% End:


The barrier works by having two counters that have to reach the threshold before the cars are released from it. Having only one such counter, one car could rush to the barrier before all of the other cars were even released. But by implementing two, that have to reach their threshold before releasing all their cars, rushing ahead to the same point, and thus stealing a ''release-token`` from one of the other cars, is impossible.


\subsection{Implementation with semaphores}
\label{sub:bar-sema}
We have implemented the barrier using the three primary functions; \code{sync()}, \code{on()}, and \code{off()} as well as an auxiliary function \code{free()}.

The snippet below shows how one of the two counters is implemented in the function \code{sync()}. Obviously, shared variables are enclosed within the statements \code{mutex.P()} and \code{mutex.V()} to ensure mutual exclusion.

If a car arrives, and the counter is below the threshold, it calls \code{barrierIncoming.P()} and waits until it gets signaled X times. When the counter reaches the threshold, \code{barrierIncoming.V()} is called X times.

The same happens for the semaphore \code{barrierLeaving} in the part of the code that has been left out.

\begin{java}
public void sync() throws Exception {
	this.mutex.P();

	if (this.active) {
		// 1st - all cars must arrive ("incoming")
		this.incomingCount++;

		if (this.incomingCount < this.threshold) { // All cars, except one
			this.mutex.V();
			this.barrierIncoming.P();
		} else { // Final car needed to start a new round
			this.free(this.incomingCount - 1, BarrierSelector.INCOMING);
			// signals the relevant barrier-semaphore (leavingCount-1) times
			this.incomingCount = 0;
			this.mutex.V();
		}
		// 2nd - all cars must leave ("leaving")
		// NB: PART LEFT OUT (This part is quite symmetrical to the first part)
		// Please see appendices for the full code*

	} else 
		this.mutex.V();
}
\end{java}
\begin{center}
* More specifically, section \ref{sub:app-barr-sync}
\end{center}


When turning on the barrier, the function \code{on()} is called, and it simply checks if the barrier is not already enabled, and if it is not, it is turned on, and the counters are set to zero.

Switching off the barrier is a bit more interesting, as the function \code{off()} is as followes (enclosed in a try-catch block and with mutual using the \code{mutex}-semaphore):

\begin{java}
if (this.active) {
	this.active = false;
	this.free(this.incomingCount, BarrierSelector.INCOMING);
	this.free(this.leavingCount+this.incomingCount, BarrierSelector.LEAVING);
	this.incomingCount = 0;
	this.leavingCount = 0;
}
\end{java}

When turning off the barrier, the semaphore for the ''arriving`` cars is signaled as many times as arriving cars has been counted - simply to release all of them. If there are any arriving cars, these will then move onto the ''leaving`` section \code{sync()}, and should be accounted for. This is why the semaphore for the leaving cars is signaled (\code{incomingCount} + \code{leavingCount}) times. Finally, the counters are reset.

\subsection{Implementation with a monitor}
\label{sub:bar-moni}
TODO!

\subsection{Implementation with a variable threshold}
\label{sub:bar-thres}
TODO!


\subsection{Testing}
\label{sub:bar-test}
TODO!