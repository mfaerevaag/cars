%%% Local Variables:
%%% mode: latex
%%% TeX-master: "../report"
%%% End:

We have successfully implemented the functionality of the steps 1-5
and ``Extra (E)'' such that it works as expected. Through this process
we have gained a good insight into the strengths and difficulties of
concurrent programming; one has to carefully analyze the problem at
hand to ensure correctness of one's approach, on the other hand, the
expressiveness is very comprehensive - hence the enormous usage of
concurrency in real life applications.

By thoroughly analyzing before implementing, model-checking and
finally running automatic tests, we believe our solutions to the
different steps work correctly. To be absolutely certain, however, we
would have to perform either formal proofs or model checking for each
feature, and verify that it represents our implementation correctly.

To us, Spin seems like an ideal choice to show correctness of one's
implementation, as modeling it in Promela is simple due to the quite
similar syntax.

Given more time, we would have made a more efficient implementation of
the field and collision avoidance. We believe it should be possible to
make a solution where the space consumption grows with the number of
cars, and not the size of the map, as it does with our current
implementation.
